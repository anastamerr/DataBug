\documentclass{article}
\usepackage[margin=1in]{geometry}
\usepackage{hyperref}

\title{DataBug AI: Intelligent Bug Triage}
\author{Team CSIS}
\date{2025}

\begin{document}
\maketitle

\section*{Overview}
DataBug AI automates bug triage by classifying incoming issues, detecting duplicates, and routing to the right teams with AI assistance. The system integrates GitHub ingestion, semantic dedupe, and an LLM assistant to reduce time-to-triage and improve consistency.

\section*{Track Alignment}
\textbf{Track 4: Bug Triage Automation} \newline
DataBug AI focuses entirely on automated bug triage workflows.

\section*{Architecture}
\begin{itemize}
  \item Bug ingestion from GitHub webhooks and optional backfill
  \item Classification for type, component, and severity
  \item Duplicate detection via semantic embeddings (Pinecone)
  \item Auto-routing to the right team
  \item LLM assistant for summaries and next actions
\end{itemize}

\section*{Impact}
\begin{itemize}
  \item Reduce triage time from minutes to seconds
  \item Improve duplicate detection accuracy
  \item Accelerate routing and resolution
\end{itemize}

\section*{Demo Flow}
\begin{enumerate}
  \item GitHub issue is created
  \item DataBug AI classifies and routes it
  \item Duplicate detector links similar issues
  \item Chat assistant summarizes impact and next steps
\end{enumerate}

\section*{References}
\begin{itemize}
  \item \href{https://arxiv.org/abs/1911.03657}{DeepTriage Paper}
  \item \href{https://www.pinecone.io/}{Pinecone Documentation}
\end{itemize}

\end{document}
